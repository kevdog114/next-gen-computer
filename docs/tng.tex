\documentclass{book}
\usepackage{amsmath}
\usepackage{listings}
\usepackage{fullpage}
\begin{document}
\noindent Kevin Schaefer \\
8-20-2014 \\
TNG - The Next Generation OS \\

\chapter{Introduction}
This paper documents the system being developed by Kevin Schaefer as a learning
exercise in computer science and electrical engineering. It consists of three
components:
\begin{enumerate}
  \item Compiler
  \item Computer
  \item Emulator
\end{enumerate}

The compiler, and associated programming language, is an exercise in writing
a complete language and compiler capable of compiling it into machine code suitable
for running on the computer or the emulator. To make it more fun, the language will
be themed after Star Trek. The high level programming language that will be developed
will be similar to the C or C++ programming languages, but will incorporate specific
elements for the computer being developed.

The computer is an exercise in computer or electrical engineering. The computer will
be developed with as many low-level electrical components as are feasible. This means
that the computer will consist of a large number of transistors and possibly logic ICs
to prevent the computer from becoming too large. This basically comes down to the fact
that building an XOR gate from transistors is pretty straight forward (and also somewhat
busy work), but using ICs that implement XOR gates helps eliminate some of the busy work.
On the other hand, certain parts of the computer, namely long-term storage, are going to
require something other than basic components.

Finally, the emulator will provide a method of testing the compiler and language without having
to copy the output onto the already built computer. This allows for faster development, and
is also pretty cool.

Depending on how all of this goes, a converter tool might also be developed to convert code
written in the custom language to C++, or vise-versa.


\chapter{TNG Architecture}
\section{Introduction}
This chapter will provide an in-depth explanation of the architecture of the
system being developed.

\section{Instruction set}
This section outlines the instruction set of the computer. \\
\begin{tabular}{|c|c|l|}
  \hline
  \textbf{Opcode} & \textbf{Hex} & \textbf{Description} \\
  \hline
  AND & 0x1 & bitwise AND \\
  OR & 0x2 & bitwise OR \\
  XOR & 0x3 & bitwise XOR \\
  INV & 0x4 & bitwise invert (not) \\
  POP & 0x5 & pop value off stack \\
  PUSH & 0x6 & push value onto stack \\
  ADD & 0x7 & Add top two values on the stack, pushing the result back on the stack \\
  SUB & 0x8 & Same as ADD, except it subtracts... \\
  MLT & 0x9 & \\
  DIV & 0xA & \\
  JPZ & 0xB & Jump if top of stack is zero \\
  JPN & 0xC & Jump if top of stack is negative \\
  LD & 0xD & Load data \\
  ST & 0xF & Store data \\
\end{tabular}

\chapter{TNG Language Specification}
\section{File types}
\begin{tabular}{|c|l|}
  \hline
  \textbf{Extension} & \textbf{Description} \\
  \hline
  .tng & Source code file \\
  .st  & Header file \\
  \hline
\end{tabular}

\section{Keywords}

\section{Sample program}
\begin{lstlisting}
**Section: Credits
  Author: Kevin Schaefer
  Date: 8-20-2014
  Package: Samples
  Email: kevin.l.schaefer1394@gmail.com
**EndSection: Credits
**Section: Code

* This is the main code section in the file

* Define a sub procedure with a parameter of type int, which returns nothing
episode printHello(int param1) space
{
  * Print a message to the screen
  write("Welcome, the parameter is ", param1);

  * Return nothing
  return space;
}

* The main entry point of the program
episode enterprise() space
{
  * Call a sub procedure
  warp printHello(1);

  * Return nothing
  return space;
}

**EndSection: Code
\end{lstlisting}

\end{document}
