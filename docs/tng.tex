\documentclass{book}
\usepackage{amsmath}
\usepackage{listings}
\usepackage{fullpage}
\title{The Next Generation Computer, Compiler, Language, and Emulator}
\date{\today}
\author{Kevin Schaefer \\ kevin.l.schaefer1394@gmail.com}
\begin{document}
\maketitle

\subsection*{Preface}
Thanks for taking time to look at my project. The basic idea behind this
project is to do something that I think is pretty cool. I basically want
to integrate aspects of Computer Science, Computer Architecture, and
Electrical Engineering all together in one project. I have no motivation
to do this project except my own desire to do something that I think is
cool and fascinating.

To anybody who is looking for help on one of the topics I touch on,
good luck and I hope my project helps.

To anybody who is looking to replicate what I have done, I am flattered
and I hope you are successful. My goal is to provide as much documentation
as you should need to build it yourself. I hope this project helps your
interest in these areas and I would love to hear your feedback.

To anybody who is looking to get out of a homework assignment by
copying my work, you won't get very far in life.

To anybody who thinks I am crazy, I don't really care; I'm doing this
anyway.

To everyone else, thanks for any interest you have in my project. I'm
not sure how much my project will benefit you, but you're welcome to
spend as much time reading through any documents or files I have provided
or to fork this project and somehow build on it further.


\tableofcontents

\chapter{Introduction}
This paper documents the system being developed by Kevin Schaefer as a learning
exercise in computer science and electrical engineering. It consists of three
components:
\begin{enumerate}
  \item Compiler
  \item Computer
  \item Emulator
\end{enumerate}

The compiler, and associated programming language, is an exercise in writing
a complete language and compiler capable of compiling it into machine code suitable
for running on the computer or the emulator. To make it more fun, the language will
be themed after Star Trek. The high level programming language that will be developed
will be similar to the C or C++ programming languages, but will incorporate specific
elements for the computer being developed.

The computer is an exercise in computer or electrical engineering. The computer will
be developed with as many low-level electrical components as are feasible. This means
that the computer will consist of a large number of transistors and possibly logic ICs
to prevent the computer from becoming too large. This basically comes down to the fact
that building an XOR gate from transistors is pretty straight forward (and also somewhat
busy work), but using ICs that implement XOR gates helps eliminate some of the busy work.
On the other hand, certain parts of the computer, namely long-term storage, are going to
require something other than basic components.

Finally, the emulator will provide a method of testing the compiler and language without having
to copy the output onto the already built computer. This allows for faster development, and
is also pretty cool.

Depending on how all of this goes, a converter tool might also be developed to convert code
written in the custom language to C++, or vise-versa.


\chapter{TNG Architecture}
\section{Introduction}
This chapter will provide an in-depth explanation of the architecture of the
system being developed \footnote{The instruction set was designed to be able to do a
wide range of things, as well as to be easy to implement in circuitry}.

This computer uses 4 bit opcodes, with 12 bit operands, for a total word size of 16 bits.
Addresses are represented with 12 bits, resulting in $2^12 = 4096$ addressable locations.

The computer maintains a system stack in memory, with the SP pointing to the top of the stack,
and an input queue, with the IQP pointing to the front of the queue. All keyboard input is stored
sequentially in the input queue, and the system stack is used for general purpose operations.

The computer also maintains an IRQ table in memory which maps interrupts to addresses, specifying
the location of the subprocedure to call when various interrupts are fired.

\section{CPU Registers}
This computer is slightly unique in that it is designed with as few registers as
possible. This is to make it easier and cheaper to build. Of course the CPU has a few
important registers, namely a PC, SP, IR, MCR, IQP, and MBR.
\subsection{PC}
  The PC, or Program Counter, stores the address of the currently executing instruction
  in memory. It is 12 bits wide.

\subsection{SP}
  The SP, or Stack Pointer, stores the address of the top of the system stack. It is
  12 bits wide.

\subsection{IR}
  The IR, or Instruction Register, stores the entire instruction and operand of the currently
  executing instruction. It is 16 bits wide (4 bit opcode + 12 bit operand).

\subsection{MCR}
The MCR, or Memory Control Register\footnote{It would ordinarily be called the Memory Address
  Register, but I needed a good way to also send the operation (read or write), so I figured a
name change to the Memory Control Register would suffice.}
  , stores the address of the location in memory to be
  accessed and the operation to be carried out. The most significant bit represents the
  memory operation, 0 represents a read and 1 represents a write, and the next 12 bits
  represent the address. The MCR is 13 bits wide.

\subsection{IQP}
  The IQP, or Input Queue Pointer, stores the address of the front of the queue where keyboard
  input to the system is stored. The IQP is 12 bits wide.

\subsection{MBR}
  The MBR, or Memory Buffer Register, stores any data received from memory. The MBR is 16 bits
  wide.

\section{Instruction set}
This section outlines the instruction set of the computer. \\
\begin{tabular}{|c|c|l|}
  \hline
  \textbf{Opcode} & \textbf{Hex} & \textbf{Description} \\
  \hline
  AND & 0x1 & bitwise AND \\
  OR & 0x2 & bitwise OR \\
  XOR & 0x3 & bitwise XOR \\
  INV & 0x4 & bitwise invert (not) \\
  POP & 0x5 & pop value off stack \\
  PUSH & 0x6 & push value onto stack \\
  ADD & 0x7 & Add top two values on the stack, pushing the result back on the stack \\
  SUB & 0x8 & Same as ADD, except it subtracts... \\
  MLT & 0x9 & \\
  DIV & 0xA & \\
  JPZ & 0xB & Jump if top of stack is zero \\
  JPN & 0xC & Jump if top of stack is negative \\
  PRT & 0xD & Print message, operand is pointer \\
  RD  & 0xE & Read data from input buffer \\
  CPU & 0xF & CPU command \\
  \hline
\end{tabular}

\subsection{AND}
  Usage: AND \textless addr\textgreater 

  The AND command performs a simple bitwise AND with the top element on the stack and
  the word pointed to by the operand. The result is pushed onto the top of the stack.

\subsection{OR}
  Usage: OR \textless addr\textgreater 

  The OR command performs a simple bitwise OR with the top element on the stack and
  the word pointed to by the operand. The result is pushed onto the top of the stack.

\subsection{XOR}
  Usage: XOR \textless addr\textgreater 

  The XOR command performs a simple bitwise XOR with the top element on the stack and
  the word pointed to by the operand. The result is pushed onto the top of the stack.

\subsection{INV}
  Usage: INV

  The INV command performs a simple bitwise NOT with the top element on the stack and
  pushes the result back onto the stack.

\subsection{POP}
  Usage: POP \textless addr\textgreater 

  The POP command removes the top element from the stack and stores it into the specified
  address location.

\subsection{PUSH}
  Usage: PUSH \textless addr\textgreater 

  The PUSH command loads a word from the specified memory location and pushes it onto
  the top of the stack.

\subsection{ADD}
  Usage: ADD \textless addr\textgreater 

  The ADD command loads a word from the specified memory location, adds it to the top element
  of the stack, and pushes the result onto the stack.

\subsection{SUB}
  Usage: SUB \textless addr\textgreater 

  The SUB command loads a word from the specified memory location, subtracts it from the top
  element of the stack, and pushes the result onto the stack.

\subsection{MLT}
  Usage: MLT \textless addr\textgreater 

  The MLT command loads a word from the specified memory location, multiplies it with the top
  element of the stack, and pushes the result onto the stack.

\subsection{DIV}
  Usage: DIV \textless addr\textgreater 

  The DIV command loads a word from the specified memory location, divides it by the top element
  of the stack, and pushes the result onto the stack.

\subsection{JPZ}
  Usage: JPZ \textless addr\textgreater
\footnote{A note on going to a specific address, you will have to force JPZ or
JPN to jump in order to achieve the same result}

  The JPZ command sets the Program Counter to the specified address if the top element of the
  stack is 0x0000.

\subsection{JPN}
  Usage: JPN \textless addr\textgreater 

  The JPN command sets the Program Counter to the specified address if the top element of the
  stack is negative.

\subsection{PRT}
  Usage: PRT \textless addr\textgreater 

  The PRT command writes data, starting at the specified address until it encounters 0x0000, to
  the display buffer.

\subsection{RD}
  Usage: RD \textless addr\textgreater 

  The RD command reads bytes from the system input buffer, pushing them onto the stack, then storing
  the count into the address specified as an operand.

\subsection{CPU}
  Usage: CPU \textless cpu\_instruction\textgreater 

  The CPU command is used to access special extended CPU instructions.

\section{Extended CPU operations}
This section describes the various extended operations the CPU can perform. These mainly are designed
to access special features of the CPU, not something a standard user-space program would need access to.
Here are the Extended CPU operations:

\begin{tabular}{|p{3cm}|p{10cm}|}
  \hline
  \textbf{Full Opcode} & \textbf{Description} \\
  \hline
  0xF0xx & Pushes the address of the IRQ handler for the specified 8 bit IRQ number \\
  0xF1xx & Pops the top element of the system stack, then stores the lesser 12 bits of that element into
           the IRQ table for the specified IRQ number \\
  0xF200 & Disables system interrupts \\
  0xF201 & Enables system interrupts \\
  0xF210 & Reads the system interrupt enable flag, and stores either 0xFFFF onto the top of the stack, or
           0x0000 onto the top of the stack \\
\end{tabular}

\section{Keyboard Input}
A computer without any means of keyboard input would be rather boring, so we are designing it to work with
a standard keyboard. Every key typed will be stored in the Input Queue. Each keypress generates a system
interrupt, so the inputted data can be processed.

\section{System Interrupts}
A number of system interrupts exist for various aspects of the system. All interrupts can be disabled via
the extended CPU operations instructions. All interrupts are stored in a table in memory and can be accessed
via the extended CPU operations. These interrupts are as follows:

\begin{tabular}{|l|l|}
  \hline
  \textbf{IRQ \#} & \textbf{Description} \\
  \hline
  0x15 & Keyboard press \\
  \hline
\end{tabular}


\chapter{TNG Language Specification}
\section{File types}
\begin{tabular}{|c|l|}
  \hline
  \textbf{Extension} & \textbf{Description} \\
  \hline
  .tng & Source code file \\
  .st  & Header file \\
  \hline
\end{tabular}

\section{Keywords}

\section{Sample program}
\begin{lstlisting}
**Section: Credits
  Author: Kevin Schaefer
  Date: 8-20-2014
  Package: Samples
  Email: kevin.l.schaefer1394@gmail.com
**EndSection: Credits
**Section: Code

* This is the main code section in the file

* Define a sub procedure with a parameter of type int, which returns nothing
episode printHello(int param1) space
{
  * Print a message to the screen
  write("Welcome, the parameter is ", param1);

  * Return nothing
  return space;
}

* The main entry point of the program
episode enterprise() space
{
  * Call a sub procedure
  warp printHello(1);

  * Return nothing
  return space;
}

**EndSection: Code
\end{lstlisting}

\end{document}
